\documentclass[12pt,a4paper,dvipdfmx]{jarticle}
\usepackage{bxpapersize}
\usepackage[dvipdfmx]{graphicx}
\usepackage{here}
\usepackage{soul}
\usepackage{ulem}

\usepackage[top=35truemm,bottom=40truemm,left=30truemm,right=30truemm]{geometry}

\usepackage{setspace}

\begin{document}

\begin{spacing}{1.5}


{\Large\textbf{$\mathrm{I}$}} リスニング問題

(解答例)

〔 I 〕

1. Hmm, I see where you're coming from, John, but I'm not entirely convinced.

2. (c)     3. (b)

\

〔 II 〕

1. multiple states simultaneously

2. Shor's algorithm (for integer factorization) , Grover's search algorithm.

3. 100 seconds

\

〔 III 〕

1. In 1451.     2. (c)     3. By using cannons.

\

\

(設問解説)




\newpage

(リスニング音声)

〔 I 〕

John:Hey guys, I was reading about stay-at-home dads recently, and I have to say, I'm really intrigued by the idea. You know, in today's world, I think it's essential to challenge those traditional gender roles. Men can be just as nurturing and capable of managing a household as women are. Plus, by having a dad stay home, it allows for more flexibility in terms of childcare and family responsibilities. It's all about equality and supporting each other's choices, you know what I mean?

\

Emily:Hmm, I see where you're coming from, John, but I'm not entirely convinced. While I'm all for breaking gender stereotypes, I can't help but feel like there's something inherently important about a man providing for his family financially. Call me old-fashioned, but I believe in the traditional breadwinner model. I worry that if men start staying at home more often, it might undermine their sense of responsibility or ambition in their careers. Plus, what message does it send to our children about gender roles and expectations?

\

Tom:Interesting points from both of you. I think there's validity to both perspectives. On one hand, I admire the idea of challenging traditional norms and embracing flexibility in family dynamics. But on the other hand, I can't ignore the potential societal implications and the importance of financial stability within a family. Maybe the key here is not to label one way as inherently better than the other, but rather to encourage open-mindedness and support for whatever works best for each individual family. After all, every family is different, and what matters most is that everyone feels respected and fulfilled in their roles.

\newpage

〔 II 〕

A quantum algorithm is an algorithm used to perform computations using a quantum computer. It may be employed to solve problems that are difficult or impossible to solve with classical computers.
	
While classical computers represent information using bits (0 or 1), quantum computers use quantum bits, or "qubits." Qubits can represent multiple states simultaneously, thanks to special properties like quantum superposition and entanglement. This enables quantum algorithms to efficiently solve more complex problems compared to classical algorithms.
	
There are various types of quantum algorithms, but some of the most famous ones include Shor's algorithm for integer factorization and Grover's search algorithm.
	
Shor's algorithm efficiently factors large composite numbers, which can be useful for breaking important cryptographic systems like RSA encryption. Its computational speed surpasses classical algorithms by a large margin.
	
Grover's search algorithm accelerates the search of unsorted data significantly faster than standard search algorithms. It speeds up the search by a square root compared to classical algorithms.
	
These are just a few examples of quantum algorithms, and as quantum computing advances, there may be the development of many more new algorithms in the future.


\newpage

〔 III 〕

Mehmed II, also known as Mehmed the Conqueror, was the 7th Sultan of the Ottoman Empire, renowned for his conquest of Constantinople and the subsequent fall of the Byzantine Empire. Mehmed II ascended to the throne in 1451 and was born in 1432. He proved to be a highly capable leader and a patron of culture and the arts.
	
During his reign, Mehmed II oversaw the reconstruction of Istanbul, the construction of palaces and mosques, and the flourishing of cultural development. His most famous achievement came in 1453 with the conquest of Constantinople. Leading the Ottoman forces, he breached the city's walls using cannons, thereby ending nearly a millennium of Byzantine rule and establishing Ottoman dominance in the eastern Mediterranean region.
	
Mehmed II passed away in 1481, and he was succeeded by his son Bayezid II. His legacy continued to exert a lasting impact on the history of the Ottoman Empire even after his death.
	

\newpage

{\Large\textbf{$\mathrm{II}$}} 『原論』より
  
(解答例)

設問(1) 点とは、数学的な位置はあるが厚み(次元でも可)がないものである。

設問(2) 幾何学的存在

設問(3) (i) (ニ)\hspace{5mm}(ii) (イ)\hspace{5mm}(iii) (ロ)

設問(4)
(i) 面
(ii) 表面とは、長さと幅を持つものである。表面は二次元の空間である。\\
        面は厚さを持たず、もし少しでも持つならば、どんなに薄くとも三次元空間である。\\
        もし表面がその中の2つの任意の点を繋ぐ直線が表面全体にある場合はそれは面と言われる。\\
        表面は、静かな水の表面或いは平らな床のように、完全に平らで均等である。


\

\

(設問解説)


\newpage

(日本語訳)


\newpage

{\Large\textbf{$\mathrm{III}$}} "Little Women"より

(解答例)

設問(1) The Civil War (American Civil War)

設問(2) 後ページ(日本語訳)にある

設問(3) (i) (ハ)\hspace{5mm}(ii) (ハ)\hspace{5mm}(iii) (イ)

設問(4) ウ→ア→キ→ク→イ→オ→カ→エ

(with as little) care and sorrow to try them as God sees fit to send

設問(5) 後ページ(日本語訳)にある

\

\

(設問解説)

\newpage

(日本語訳)





\end{spacing}
  
\end{document}
