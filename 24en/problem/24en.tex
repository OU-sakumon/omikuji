\documentclass[12pt,a4paper,dvipdfmx]{jarticle}
\usepackage{bxpapersize}
\usepackage[dvipdfmx]{graphicx}
\usepackage{here}
\usepackage{soul}
\usepackage{ulem}

\usepackage[top=35truemm,bottom=40truemm,left=30truemm,right=30truemm]{geometry}


\usepackage{setspace}

\begin{document}

\setcounter{page}{1}

\begin{spacing}{1.5}

% 1/4

{\Large\textbf{$\mathrm{I}$}} リスニング問題です。いずれもQRコードを読み取って,問題音声をダウン

 ロードしなさい。再生を押すとただちに文章が読まれます。

 なお,作問にあたって以下のサイトを利用しました。

 ・ChatGPT3.5 (https://chat.openai.com)

 ・Genny by LOVO (https://genny.lovo.ai/my-workspace/project)


% 1/4 -1

\vspace{7mm}

〔 $\mathrm{I}$ 〕 John,Emily,Tomが順に話す。

\vspace{3mm}
\begin{figure}[H]
\centering
\includegraphics[scale=0.2]{conversation.jpeg}
\end{figure}
\vspace{-1mm}

1. Write the first sentence Emily says.

2. What topic are they discussing? Choose the best option.

 (a) gender inequality in work places

 (b) how to maintain a good relationship with their spouse

 (c) if they agree or disagree with the idea of stay-at-home dads

 (d) the issue of the traditional breadwinner model

3. All of the three seem to think that

 (a) every partner should share the housework.

 (b) it is generally expected for men to work in current society.

 (c) the idea that men should earn more than women is obsolete.

 (d) women should pursue fulfilling jobs.
 

% 1/4 -2
 
\newpage

〔 $\mathrm{II}$ 〕

\vspace{3mm}
\begin{figure}[H]
\centering
\includegraphics[scale=0.2]{quantum.jpeg}
\end{figure}
\vspace{-1mm}

1. Fill in the blanks with the words used in the audio. 

 As opposed to classic binary bits, qubits can represent〔   〕.

\vspace{2mm}

2. Provide two examples of quantum algorithms introduced in the audio.

\vspace{2mm}

3. How many steps would it take Grover's search algorithm to finish a search

 of unsorted data that classic algorithms need 10,000 steps?

% 1/4 -3

〔 $\mathrm{III}$ 〕

\vspace{3mm}
\begin{figure}[H]
\centering
\includegraphics[scale=0.2]{mehmed.jpeg}
\end{figure}
\vspace{-1mm}

1. When did Mehmed II became the sultan, the emperor of the Ottoman Empire?

\vspace{2mm}

  2. Which is the exact existence duration of the Byzantine Empire?

 (a) 27 BC $-$ 1453 AD  \hspace{2mm}  (b) 27 BC $-$ 1481 AD

 (c) 395 AD $-$ 1453 AD    (d) 395 AD $-$ 1481 AD

\vspace{2mm}

  3. How did Mehmed II conquer the Byzantine Empire?


\newpage


  {\Large\textbf{$\mathrm{II}$}}  高校数学では「ユークリッドの互除法」として、大学の論理学・幾何・代数でも「ユークリッド空間」「ユークリッド距離」等の名称で出てくる「ユークリッド」とは、古代ギリシアの数学者「エウクレイデス」の形容詞形(古代ギリシア語文法用語では属格曲用という)を、英語風に発音したものである。\\
  以下の文章は、そんな彼が著したとされる『原論』の現代英語訳の冒頭である。本文を読んで後の問いを答えよ。

  \vspace{7mm}
  \begin{center}
    \LARGE
    \textbf{BOOK I.}
  \end{center}
  \begin{center}
    \textbf{THEORY OF ANGLES, TRIANGLES, PARALLEL LINES, AND PARALLELOGRAMS.}
  \end{center}
\begin{center}
  \rule{\textwidth}{0.4pt}
\end{center}

\begin{center}
  \textbf{The Point.}
\end{center}
\vspace{2mm}
I. $_{\rm(A)}$\ul{A point is that which has position but not dimensions}.
A $_{\rm(B)}$\ul{geometrical magnitude} which has three dimensions, that is, length, breadth, and thickness, is a solid; that which has two dimensions, such as length and breadth, is a surface; and
that which has but one dimension is a line. But a point is neither a solid, nor a surface, nor
a line; hence it has no dimensions—that is, it has neither length, breadth, nor thickness. \\

\begin{center}
  \textbf{The Line.}
\end{center}
\vspace{2mm}
II. A line is length without breadth.
A line is space of one dimension. If it had any breadth, no matter how small, it would
be space of two dimensions; and if in addition it had any thickness it would be space of three
dimensions; hence a line has neither breadth nor thickness. \\

III. The intersections of lines and their extremities are points. \\

IV. A line which lies evenly between its extreme points is called a straight or right line, such as AB.
If a point move without changing its direction it will describe a right line. The direction in
which a point moves in called its “sense.” If the moving point continually changes its direction
it will describe a curve; hence it follows that only one right line can be drawn between two
points. The following Illustration is due to Professor Henrici:— “If we suspend a weight by a
string, the string becomes stretched, and we say it is straight, by which we mean to express
that it has assumed a peculiar definite shape. If we mentally abstract from this string all
thickness, we obtain the notion of the simplest of all lines, which we call a straight line.”

\vspace{2mm}

\vspace{2mm}
\begin{center}
  \textbf{The Plane.}
\end{center}

\ul{V. A surface is that which has length and breadth.
A surface is space of two dimensions. It has no thickness, for if it had any, however small,
it would be space of three dimensions.} \\

\ul{VI. When a surface is such that the right line joining any two arbitrary
points in it lies wholly in the surface, it is called a plane.
A plane is perfectly flat and even, like the surface of still water, or of a smooth floor.}—
Newcomb.
  
\

設問(1) 下線部(A)に関して、この文脈におけるpositionとdeimensionの違いが分かるように対比させつつ和訳せよ。

\

設問(2) 下線部(B)\underline{geometrical magnitude}を、前後の文脈から判断して適切な日本語に訳せ。

\


設問(3) 下線部(i)~(iii)の語句の本文中での意味に最も近いものを,(イ)~(ニ)

    から1つ選びなさい。


    (i)\hspace{5mm}hence

    
     (イ) initially\hspace{22mm}(ロ) inevitably

     (ハ) meanwhile\hspace{17mm}(ニ) consequently


    (ii)\hspace{5mm}breath

    
     (イ) width\hspace{25.5mm}(ロ) length

     (ハ) strength\hspace{21.0mm}(ニ) health


    (iii)\hspace{5mm}string

    
     (イ) number\hspace{22.5mm}(ロ) cord

     (ハ) text\hspace{29.0mm}(ニ) cup

\

設問(4) 

(i) 下線部(C)において、planeは通常とは異なる特別な意味になる。\\
\hspace{10mm}それまでのThe Point., The Line., をヒントに、日本語で適切な表現に直せ。

(ii) (i)を踏まえ、下線部bを日本語に直せ。


\newpage


{\Large\textbf{$\mathrm{III}$}} 以下の文章を読んで後の問いを答えよ。

\vspace{7mm}

【あらすじ】

貧乏なMarch家は父,母,それと4人の姉妹で構成されている。父は$_{\rm(A)}$\underline{とある戦争}に牧師として従軍し,一年間帰らないことになっている。その父が戦場から家族(母)に宛てた手紙にはこうあった。

“Give them all of my dear love and a kiss. Tell them I think of them by day, pray for them by night, and find my best comfort in their affection at all times. A year seems very long to wait before I see them, but remind them that while we wait we may all work, so that these hard days need not be wasted. $_{\rm(B)}$\ul{I know they will remember all I said to them, that they will be loving children to you, will do their duty faithfully, fight their bosom enemies bravely, and conquer themselves so beautifully that when I come back to them I may be fonder and prouder than ever of my little women.}”

``Little Women"は全文を通して,姉妹が父の教えを守り成長していく一年を描いた作品であるが,問題にするには長大すぎる。今回は,4姉妹の長女Megが裕福なMoffat家に招待されたパーティー後の一幕を切り取った。パーティーの中で彼女は自身の短所であった「見栄っ張り」を克服してゆく。まさに核となるお話なのは承知でも問題文として取り上げられなかったのが残念である。

さて,Megはこのパーティーでとあるゴシップを耳にする。それというのが,「この頃March家は良家の隣人Laurieと親睦を深めているようだが,それはMarch夫人(Megたちの母)が娘の誰かをLaurieと結婚させてやろうという料簡らしい」といった内容であった。帰宅してから次女のJoと母に伝えるとJoは激昂し,母も立腹しながら,娘に嫌な話を聞かせたことにやるせなさを感じて謝罪する。もやもやした心のMegはためらいがちにも,母にそうした「料簡」があったか問うことにしたところから話しは始まる。

\newpage

``Mother, do you have `plans', as Mrs. Moffat said?" asked Meg bashfully.

``Yes, my dear, I have a great many, all mothers do, but mine differ somewhat from Mrs. Moffat's, I suspect. I will tell you some of them, for the time has come when a word may set this romantic little head and heart of yours right, on a very serious subject. You are young, Meg, but not too young to understand me, and mothers' lips are the fittest to speak of such things to girls like you. Jo, your turn will come in time, perhaps, so listen to my `plans' and help me carry them out, if they are good."
Jo went and sat on one arm of the chair, looking as if she thought they were about to join in some very $_{\rm(i)}$\ul{solemn} affair. Holding a hand of each, and watching the two young faces wistfully, Mrs. March said, in her serious yet cheery way...

``I want my daughters to be beautiful, accomplished, and good. To be admired, loved, and respected. To have a happy youth, to be well and wisely married, and to lead useful, pleasant lives, with as little $_{\rm(C)}$\ul{[ ア and sorrow / イ as / ウ care / エ fit / オ God / カ sees / キ to / ク try them ] }to send. To be loved and chosen by a good man is the best and sweetest thing which can happen to a woman, and I sincerely hope my girls may know this beautiful experience. It is natural to think of it, Meg, right to hope and wait for it, and wise to prepare for it, so that when the happy time comes, you may feel ready for the duties and worthy of the joy. My dear girls, I am ambitious for you, but not to have you make a dash in the world, marry rich men merely because they are rich, or have splendid houses, which are not homes because love is wanting. $_{\rm(D)}$\ul{Money is a needful and precious thing, and when well used, a noble thing, but I never want you to think it is the first or only prize to strive for. I'd rather see you poor men's wives, if you were happy, beloved, contented, than queens on thrones, without self-respect and peace.}"

``Poor girls don't stand any chance, Belle says, unless they put themselves forward," sighed Meg.

``Then we'll be old maids," said Jo $_{\rm(ii)}$\ul{stoutly}.

``Right, Jo. Better be happy old maids than unhappy wives, or unmaidenly girls, running about to find husbands," said Mrs. March decidedly. ``Don't be troubled, Meg, poverty seldom $_{\rm(iii)}$\ul{daunts} a sincere lover. Some of the best and most honored women I know were poor girls, but so love-worthy that they were not allowed to be old maids. Leave these things to time. Make this home happy, so that you may be fit for homes of your own, if they are offered you, and contented here if they are not. One thing remember, my girls. Mother is always ready to be your confidant, Father to be your friend, and both of us hope and trust that our daughters, whether married or single, will be the pride and comfort of our lives."

``We will, Marmee, we will!" cried both, with all their hearts, as she bade them good night.

\vspace{-5mm}\begin{center}
(Louisa May Alcott. 1868. Little Women)
\end{center}\vspace{-5mm}

\

設問(1) ``Little Women"は半ば作者の自叙伝であり,実際,Joは自身を投影

    した人物である。さて,下線部(A)も実際にあった戦争だが,それは何

    か。英語で答えよ。

\

設問(2) 下線部(B)を日本語で表せ。

\

設問(3) 下線部(i)~(iii)の語句の本文中での意味に最も近いものを,(イ)~(ニ)

    から1つ選びなさい。


    (i)\hspace{5mm}solemn

    
     (イ) complicated\hspace{20mm}(ロ) depressing

     (ハ) grave\hspace{32mm}(ニ) scandalous


    (ii)\hspace{5mm}stoutly

    
     (イ) admonishingly\hspace{15.5mm}(ロ) jokingly

     (ハ) powerfully\hspace{22.75mm}(ニ) weakly



\

    (iii)\hspace{5mm}daunts

    
     (イ) baffles\hspace{30.5mm}(ロ) encourages

     (ハ) suffocates\hspace{24.5mm}(ニ) strongthens

\

設問(4) 下線部(C)を正しく並べ替えよ.ただし,「ア→イ→…」と書け.


設問(5) 下線部(D)を日本語で表せ.





\newpage

{\Large\textbf{$\mathrm{IV}$}} 以下の文章について後の問いに答えよ。


\vspace{7mm}

\underline{すべて世のありにくきこと、わが身とすみかとの、はかなくあだなるさま$^{1}$かく}

\hspace{-4mm}
\underline{のごとし。いはむや$^{2}$所により、身のほどにしたがひて、心をなやますこと、あげ}

\hspace{-4mm}
\underline{てかぞふべからず。}もしおのづから身かずならずして、$^{3}$權門のかたはらに居るものは深く悦ぶことあれども、大にたのしぶにあたはず。なげきある時も聲をあげて泣くことなし。進退やすからず、たちゐにつけて恐れをのゝくさま、たとへば、雀の鷹の巣に近づけるがごとし。もし貧しくして富める家の隣にをるものは、朝夕すぼき姿を耻ぢてへつらひつゝ出で入る妻子、僮僕のうらやめるさまを見るにも、富める家のひとのないがしろなるけしきを聞くにも、心念々にうごきて時としてやすからず。もしせばき地に居れば、近く炎上する時、その害をのがるゝことなし。もし邊地にあれば、往反わづらひ多く、盜賊の難はなれがたし。いきほひあるものは貪欲ふかく、ひとり身なるものは人にかろしめらる。寶あればおそれ多く、貧しければなげき切なり。人を頼めば身他のやつことなり、人をはごくめば心恩愛につかはる。世にしたがへば身くるし。またしたがはねば狂へるに似たり。いづれの所をしめ、いかなるわざをしてか、しばしもこの身をやどし$^{4}$玉ゆらも心をなぐさむべき。

\vspace{7mm}

$1$ 様々な自然災害とその被害のことを指している。特にこの文章の直前では自ら

 が体験した大地震についての記述がある。

$2$ 環境や境遇によって

$3$ 権力のある家の隣に住む者

$4$ 少しの間

\vspace{-5mm}\begin{center}
(鴨長明,方丈記 青空文庫より)
\end{center}\vspace{-5mm}

\newpage

設問(1) 下線部の意味を英語で表しなさい。

\

設問(2) 本文の要旨を英語で表しなさい。

\

設問(3) 昨今あげられる「古文不要論」について,あなたの考えを80語以上の

    英語で述べなさい。

\end{spacing}

\end{document}